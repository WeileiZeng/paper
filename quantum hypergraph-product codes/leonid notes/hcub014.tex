\def\rank{\mathop{\rm rank}}
\def\wgt{\mathop{\rm wgt}}
\def\lc{\mathop{\rm lc}}
\documentclass[aps,prb,12pt,tightenlines,%
notitlepage,longbibliography]{revtex4-1}

\usepackage{hyperref}
\usepackage{graphicx}
\usepackage{amsthm}
\usepackage{amsfonts}
\usepackage{array}
\usepackage{enumitem}
\newtheorem{theorem}{Theorem}
\newtheorem{note}[theorem]{Note}
\newtheorem{corollary}[theorem]{Corollary}
\newtheorem{lemma}[theorem]{Lemma}
\newtheorem{definition}[theorem]{Definition}
%\advance\textheight by -3.2in
\begin{document}
\title{Higher-dimensional hypergraph-product codes}
\date{August 15, 2018} \author{Leonid P.\ Pryadko}
\address{UCR}
\author{Weilei Zeng}
\address{UCR}
\begin{abstract}
  A higher-dimensional version of the quantum hypergraph-product
  ansatz is described.  Most important feature are the sharp lower and
  upper bounds for the minimum distance.
\end{abstract}
\maketitle

\section{Introduction}

The construction is useful for (a) analyzing repeated measurement in a
stabilizer code in the problem of fault-tolerant (FT) quantum error
correction (e.g., surface
code\cite{Dennis-Kitaev-Landahl-Preskill-2002}, or more general LDPC
code\cite{Kovalev-Pryadko-FT-2013}, (b) related problem of single-shot
error correction\cite{Campbell-2018}, (c) analysis and generalization
of transformations between QECCs, like the quantum code enlargement
trick by Hastings [ref?], and (d) construction of asymmetric quantum
CSS codes optimized for operation where $X$ and $Z$ channels are
strongly asymmetric\cite{Ioffe-Mezard-2007,Evans-2007,Stephens-2008,%
  Aliferis-Preskill-2008,sarvepalli-2009,Tuckett-Bartlett-Flammia-2018}.

Origin: (a) Higher-dimensional toric codes [\textbf{ref}?] and (b)
Hypergraph-product codes\cite{Tillich-Zemor-2009}.  Present
construction generalizes both.

\section{Construction overview}

We  want to generalize the QHP construction\cite{Tillich-Zemor-2009} to
produce longer chain complexes from two, three, etc.\ arbitrary size
binary matrices.  Here are some examples of intended construction.
\begin{itemize}
\item Original QHP construction can be interpreted as a length-two
  chain complex $ {\cal K}(A_1,A_2)$:
  $\{0\}\leftarrow C_0\stackrel{A_1}\leftarrow C_1\stackrel{A_2}\leftarrow
  C_2\leftarrow \{0\}$ with boundary operators such that $A_1A_2=0$
  (below, the indices above and to the left of the double lines label
  the corresponding sectors),
  \begin{equation}
A_1=\left(
  \begin{array}[c]{c||c|c}
& x & y\\ \hline\hline 
\cdot & H_1\otimes E & E\otimes H_2
  \end{array}
\right), \quad A_2=
\left(
  \begin{array}[c]{c||c}
   &  xy \\ \hline\hline 
x&    E\otimes H_2\\
y&    H_1\otimes E
  \end{array}
\right).\label{eq:two-chain-complex-QHP}
\end{equation}
Here $H_i$, $i\in \{1,2\}$ are binary matrices and $E$ denotes
identity matrices of the sizes needed to make the block rows and block
columns of matching sizes.
\item The following three-chain complex ${\cal K}(B_1,B_2,B_3)$
  $\{0\}\leftarrow C_0\stackrel{B_1}\leftarrow C_1\stackrel{B_2}\leftarrow
  C_2\stackrel{B_3}\leftarrow C_3$ with boundary operators
  $B_1B_2=0$, $B_2B_3=0$ gives a generalization of the 3D toric codes'
  construction:
  \begin{eqnarray*}
    B_1&=&\left(\cdot \| H_1\otimes E\otimes E|E\otimes H_2\otimes E|E\otimes
           E\otimes H_3\right)=
           \left(A_1\otimes E|E\otimes H_3\right),\\
    B_2&=&\left(
           \begin{array}[c]{c||c|c|c}
             & xy & xz & yz \\ \hline\hline 
             x& 
                E\otimes H_2\otimes E  & E\otimes E\otimes H_3 &  \\ \hline 
             y&   H_1\otimes E\otimes E& & E\otimes E\otimes H_3 \\  \hline 
             z&  &  H_1\otimes E\otimes E& E\otimes H_2\otimes E
           \end{array}\right)
                                           =\left(
                                           \begin{array}[c]{c|c}
                                             A_2\otimes E & E\otimes H_3\\
                                             \hline     
                                                          &  A_1\otimes E
                                           \end{array}
                                                            \right)
                                                            ,\\
    B_3&=&\left(
           \begin{array}[c]{c||c}
             &     xyz \\ \hline\hline 
             xy &    E\otimes E\otimes H_3   \\ \hline 
             xz &   E\otimes H_2\otimes E  \\  \hline 
             yz &    H_1\otimes E\otimes E
           \end{array}\right)=\left(
                  \begin{array}[c]{c}
                    E\otimes H_3   \\ \hline 
                    A_2\otimes E
                  \end{array}
    \right).
  \end{eqnarray*}
\end{itemize}
The simple rule is that the matrix $H_1$, $H_2$, or $H_3$ is placed at
the position whose column label differs from the row label by $x$,
$y$, or $z$, respectively.  The construction has an obvious
generalization to higher dimensions.  In the following, it is more
convenient to use the following (equivalent) recursive definition:

\begin{definition}[Dimensional extension of a chain complex]
  \label{def:dim-ext} Given an $(m-1)$-chain complex
  $\mathcal{A}\equiv\mathcal{K}(A_1,\ldots,A_{m-1})$:
  $\{0\}\leftarrow C_0\stackrel{A_1}\leftarrow
  C_1\stackrel{A_2}\leftarrow C_2\ldots \stackrel{A_{m-1}}\leftarrow
  C_{m-1}\stackrel{0_{n_{m-1}\times 0}}\leftarrow \{0\}$ with
  non-trivial boundary operators $A_j$, $j=1,\ldots,m-1$, such that
  $A_jA_{j+1}=0$, and an $r\times c$ binary matrix $P$, the extended
  $m$-chain complex $\mathcal{B}\equiv\mathcal{K}(B_1,\ldots ,B_m)$ is
  defined by the boundary operators
  \begin{eqnarray}
    \label{eq:B1}
    B_1&=&(A_1\otimes E_{r}|E_{n_0}\otimes P),\quad 
           B_2=\left(
           \begin{array}[c]{c|c}
             A_2\otimes E_r&E_{n_1}\otimes P\\ \hline
                           & A_1\otimes E_c
           \end{array}\right), \ldots\\ 
    \label{eq:Bj}
    B_j&=&\left(
           \begin{array}[c]{c|c}
             A_j\otimes E_r&E_{n_{j-1}}\otimes P\\ \hline
                           & A_{j-1}\otimes E_c
           \end{array}\right), \quad \ldots\\
    B_{m-1}&=&\left(
               \begin{array}[c]{c|c}
                 A_{m-1}\otimes E_r&E_{n_{m-2}}\otimes P\\ \hline
                                   & A_{m-2}\otimes E_c
               \end{array}\right),\\ \label{eq:Blast}
    B_m&=&\left(
           \begin{array}[c]{c}
             E_{n_{m-1}}\otimes P\\  \hline        A_{m-1}\otimes E_c
           \end{array}\right).
  \end{eqnarray}
  Here $E_r\equiv E(r)$ denotes the $r\times r$ identity matrix, and
  the original linear spaces $C_i$, $i\in\{0,\ldots,m-1\}$, have
  dimensions $n_i$ (so that $A_i$ is an $n_{i-1}\times n_i$ binary
  matrix), with the additional convention $n_{j}=0$ for $j<0$ and
  $j\ge m$.  The dimension of thus defined $j$-th level extended
  linear space $\mathcal{C}_j'$ is $n_j'=n_j r+n_{j-1} c$,
  $j\in\{0,\ldots,m\}$.
\end{definition}
The constructed matrices trivially satisfy the correct orthogonality
conditions.  The original non-recursive definitions can be recovered
by starting with the two-chain complex
(\ref{eq:two-chain-complex-QHP}) and extending it sequentially with
the help of matrices $H_3$, $H_4$, etc.

Let us assume that the rank of the $j$-th homology group
$\mathcal{H}_j\equiv \mathcal{H}_j(\mathcal{A})$ is
$k_j=\rank(\mathcal{H}_j)$, and the corresponding \emph{distance}, the
minimum weight of a homologically non-trivial cycle in $\mathcal{H}_j$
is $d_j\ge1$, $1\le j<m$.  Here we use the convention that
$d_j=\infty$ if $k_j=0$.  The distance $d_j$ can be also expressed as
the (left or $z$) distance of a quantum CSS code with generators
$G_x=A_j$ and $G_z=A_{j+1}^T$, denoted as
$\mathcal{Q}(A_j,A_{j+1}^T)$, that is,
$$d_j(\mathcal{A})\equiv d_j(A_j,A_{j+1}^T) =\min_{\mathbf{e}\in\mathcal{C}_{A_j}^\perp\setminus\mathcal{C}_{A_{j+1}^T}}\wgt
\mathbf{e}.$$ 
Here ${\cal C}_G$ and ${\cal C}^\perp_H$  denote the
binary codes with the generator matrix $G$ and the check matrix $H$,
respectively. 

The parameters of the CSS code $\mathcal{Q}(A_j,A_{j+1}^T)$ are thus
$[[n_j,k_j,\min(d_j,\tilde{d}_{j})]]$, where the distances in the
corresponding co-chain complex generated by transposed matrices
$A_j^T$, with the list taken in the opposite order,
$$
\tilde{\mathcal{A}}\equiv \{0\}\leftarrow
\mathcal{C}_{m-1}\stackrel{A_{m-1}^T}\leftarrow \ldots
\stackrel{A_{1}^T}\leftarrow \mathcal{C}_0\leftarrow \{0\}, 
$$
are denoted with the tilde, $\tilde{d}_j\equiv d(A_{j+1}^T,A_j)$.

For the following, it is convenient to extend the definitions of the
distance to also apply for the portions $\{0\}\leftarrow{C}_0$, as
well as the portion $C_{m-1}\leftarrow\{0\}$.  That is, we would like
to formally define quantum CSS codes ${\cal Q}(0_{0\times n_0},A_1^T)$
and ${\cal Q}(A_{m-1},0_{n_{m-1}\times 0})$.  In the former case, we
are looking at binary vectors in $\mathbb{F}_{n_0}$ orthogonal to a
zero-row matrix (all $2^{n_0}$ vectors), except those that are
linearly dependent with the rows of $A_0^T$.  It is easy to check that
the minimum distance of such a code is $1$, except when $A_0^T$ is a
full-row-rank matrix, in which case ${\cal Q}(0,A_0^T)$ is empty.  By
convention, in the latter case, we are going to set $d=\infty$.  Thus,
we have \framebox{$d_0\equiv d(0,A_0)=k_0>0?1:\infty$,} that is,
distance of the code ${\cal Q}(0,A_0)$ is one unless $n_0=\rank A_0$,
in which case it is infinite.   

The situation is even simpler in the case of
${\cal Q}_m={\cal Q}(A_{m-1},0)$, which is just a classical binary
code with the check matrix $A_{m-1}$ of dimension $n_{m-1}\times n_m$
(the zero matrix does not remove any vectors from
${\cal C}_{A(m-1)}^\perp$).  The parameters of this classical code
with the check matrix $A_{m-1}$ are
$[n_{m-1},n_{m-1}-\rank A_{m-1},d_m]$.  Again, by convention, the
distance $d_m$ is infinite if the code is empty, that is, if $A_{m-1}$
is a full-column-rank matrix.


Main result of this work are the exact parameters of the
dimensionally-extended chain complex $\mathcal{B}$, expressed in terms
of those of the original chain complex $\mathcal{A}$ and the
parameters of the two binary codes with parity check matrices $P$ and
$P^T$, respectively: $\mathcal{C}_P^\perp=[c,\kappa,\delta]$ and
$\mathcal{C}_{P^T}^\perp=[r,\tilde{\kappa},\tilde{\delta}]$.  Here $P$
is an $r\times c$ binary matrix with $\rank P=u$, so that $\kappa=c-u$
and $\tilde{\kappa}=r-u$; and the distances $\delta\ge1$,
$\tilde{\delta}\ge1$.  Throughout this work the
infinite-distance-empty-code convention is used: $\delta=\infty $ if
$\kappa=0$, and $\tilde{\delta}=\infty$ if $\tilde{\kappa}=0$.

With these definitions, the parameters of the dimensionally-extended
$m$-chain complex $\mathcal{B}$ (see Definition \ref{def:dim-ext}) are
given by the following
\begin{theorem}
  \label{th:results}
  The dimension of the space $\mathcal{C}_j'$ is
  \framebox{$n_j'=n_j r+n_{j-1}c$,} the rank of the $j$\,th homology
  group $\mathcal{H}_j(\mathcal{B})$ is
  \framebox{$k_j'=k_j\tilde{\kappa}+ k_{j-1}\kappa$,} with the minimum
  distance [weight of the smallest homologically non-trivial cycle in
  $\mathcal{H}_j(\mathcal{B})]$\ \ \framebox{$d_j'=d_{j-1}\delta$ if
    $\tilde{\kappa}=0$, otherwise $d_j'=\min(d_j,d_{j-1}\delta)$.}
\end{theorem}

This theorem combines the results of Lemma \ref{th:rankBj}, Theorem
\ref{th:upper-distance-bnd} (upper distance bounds), and Theorem
\ref{th:lower-distance-bnd} (lower distance bound) below.  The value
of $n_j'$ directly follows from the Definition \ref{def:dim-ext}.
\section{Matrix ranks and dimensions of CSS codes.}
Any pair of adjacent matrices $A_{j}$ and $A_{j+1}^T$ can obviously be
used as generators of a quantum CSS code
${\cal Q}_j\equiv {\cal Q}(A_{j},A_{j+1}^T)$, which we assume to encode
$k_j$ qubits in $n_j$.  Alternatively, $k_j$ is the rank of the
$j$\,th homology group in the original chain complex $\mathcal{A}$.

What are the parameters $[[n_j',k'_j,d'_j]]$
of the codes generated by dimensionally extended matrices, e.g.,
${\cal Q}_j'\equiv {\cal Q}(B_{j},B_{j+1}^T)$?  

We already know the block length {$n_j'=n_j r+n_{j-1}c$.}  To
find $k_j'$, denote the original ranks $s_j=\rank A_j$,
$u\equiv \rank P$, and prove:
\begin{lemma}
  \label{th:rankBj}
  Given $A_{j-1}A_{j}=0$, the  binary matrix
  $$    B_j=\left(
    \begin{array}[c]{c|c}
      A_j\otimes E_r& E_{n_{j-1}}\otimes P\\ \hline%\hrule
                    & A_{j-1}\otimes E_c
    \end{array}\right)
  $$
  has the rank $s_j'\equiv \rank B_j=s_j (r-u)+s_{j-1}(c-u)+n_{j-1} u$.
\end{lemma}
%\end{document}
\begin{proof}
  Start by computing the ranks of the upper and lower row blocks using
  the trick from Ref.~\onlinecite{Tillich-Zemor-2009}.  Namely, for
  each block, use
  row transformations to form distinct zero combinations.  The
  corresponding count for the upper row block is
$$\lc B_j'=(n_{j-1}-s_j)(r-u),$$
and for the lower row block, trivially,
$\lc B_j''=(n_{j-2}-s_{j-1}) c$.  This gives
$\rank B_j'=s_j r+n_{j-1}u -s_j u$, and $\rank B_j''=c s_{j-1}$.  For
matrices of this form, the number of  zero linear
combinations that involve both row blocks is just the product of the
ranks in the right blocks, 
$$
\lc(B_j',B_j'')=s_{j-1}u.
$$
Overall, we get the stated rank of $B_j$,
\begin{eqnarray*}
  s_j'&=& \rank B_j=(n_{j-1}r+n_{j-2}c)-(n_{j-1}-s_j)(r-u
          )-(n_{j-2}-s_{j-1})c-s_{j-1} u \\ 
      &=& n_{j-1}r +n_{j-2}c -n_{j-1}r+r s_j+n_{j-1} u -s_j u
          -cn_{j-2}+cs_{j-1}-s_{j-1}u \\  
      &=& s_j(r-u )+s_{j-1} (c-u )+n_{j-1}u .      
\end{eqnarray*}
\end{proof}
The same result can also be obtained considering column blocks.
Check:
\begin{eqnarray*}
  s_j'&=&n_j r+n_{j-1}c-(n_{j-1}-s_{j-1})(c-u)-(n_j-s_j)r-s_ju\\
  &=& n_{j-1}u+c s_{j-1}-s_{j-1}u+rs_j-s_j u= 
\framebox{$s_j(r-u )+s_{j-1} (c-u )+n_{j-1}u .$}
\end{eqnarray*}
These expressions give 
\begin{eqnarray}
  \nonumber
  k_j'&=& n_{j}'-\rank B_{j}-\rank B_{j+1}\\   \nonumber
      &=& n_j r+n_{j-1}c
          - s_{j  }(r-u)-s_{j-1}(c-u)-n_{j-1}u 
          - s_{j+1}(r-u)-s_{j}(c-u)-n_{j}u 
\\ \nonumber 
  &=& (n_j-s_{j}-s_{j+1})(r-u)  +(n_{j-1}-s_{j-1}-s_{j})(c-u) \\
  & =& k_j \tilde{\kappa}+k_{j-1}\kappa,
\label{eq:kj-prime}
\end{eqnarray}
where $\kappa\equiv c-u$ and $\tilde{\kappa}\equiv r-u$ respectively are the
dimensions of the binary codes using $P$ and $P^T$ as check matrices.

For the length-two chain complex (\ref{eq:two-chain-complex-QHP}) this
gives (setting $s_{0}=0$): 
\begin{eqnarray}
  k_0^{(2)}&=&r_1r_2-\rank
               B_1=(r_1-u_1)(r_2-u_2)=\tilde{\kappa}_1\tilde{\kappa}_2,\\ 
  k_1^{(2)}&=&(c_1-u_1)(r_2-u_2)+(r_1-u_1)(c_2-u_2)=\tilde{\kappa}_1\kappa_2+\kappa_1\tilde{\kappa}_2,\label{eq:k-two}
  \\
  k_2^{(2)}&=&c_1c_2-\rank B_2=(c_1-u_1)(c_2-u_2)=\kappa_1\kappa_2,
\end{eqnarray}
where the $k_0$ and $k_2$ correspond to classical codes with check
matrices $B_1^T$ and $B_2$, respectively.  The value of $k_1^{(2)}$
coincides with the result in Ref.~\onlinecite{Tillich-Zemor-2009}.
For length-3 chain complex one gets, explicitly:
\begin{eqnarray*}
  k_0^{(3)}&=&\tilde{\kappa}_1\tilde{\kappa}_2\tilde{\kappa}_3,\\
  k_1^{(3)}&=&{\kappa}_1\tilde{\kappa}_2\tilde{\kappa}_3+\tilde{\kappa}_1{\kappa}_2\tilde{\kappa}_3+\tilde{\kappa}_1\tilde{\kappa}_2{\kappa}_3,\\
  k_2^{(3)}&=&{\kappa}_1{\kappa}_2\tilde{\kappa}_3+{\kappa}_1\tilde{\kappa}_2{\kappa}_3+\tilde{\kappa}_1{\kappa}_2{\kappa}_3,\\
  k_3^{(3)}&=&{\kappa}_1{\kappa}_2{\kappa}_3.
\end{eqnarray*}
Generally, $k_j^{(m)}$ is a sum of all the products of distinct
$\kappa_s$, $s\in\{1,\ldots,m\}$ with all possible placements of
$(m-j)$ tildes.

\section{Bounds on codes' distances}

For a CSS code with binary generator matrices $P$ and $Q$, such that
$PQ^T=0$, consider two distances different by the order of the
generators:
$$
 d(P,Q)=\min_{\mathbf{x}\in {\cal C}_{P}^\perp\setminus {\cal
     C}_{Q}} \wgt \mathbf{x},\quad 
 d(Q,P)=\min_{\mathbf{x}\in {\cal C}_{Q}^\perp\setminus {\cal
     C}_{P}} \wgt \mathbf{x},
$$
so that the actual CSS code distance is the minimum of the two.  We
will use the convention\cite{Tillich-Zemor-2009} that an empty code
has infinite distance, which is the same as stating
$\min(\emptyset)=\infty$.  Notice that if one takes the right matrix
zero, one can also get the distances of the classical binary code
associated with the matrix $P$ (dimensions $r\times c$),
$$
d_P=d(P,0_{0\times c}).
$$ 
In any case, the distance of the code (quantum or classical, empty or
not) cannot be zero.

Again, consider an $(m-1)$-chain complex
$$\mathcal{A}=\{0\}\stackrel{0_{0\times n_0}}\leftarrow C_0\stackrel{A_1}\leftarrow
C_1\leftarrow\ldots\leftarrow C_{m-2}\stackrel{A_{m-1}}\leftarrow
C_{m-1}\stackrel{0_{ n_{m-1}\times 0}}\leftarrow \{0\},$$ and its
associated dimensionally-extended $m$-chain complex $\mathcal{B}$, see
Definition \ref{def:dim-ext}.  Let the CSS code
${\cal Q}(A_j,A_{j+1}^T)$ have parameters $[[n_j,k_j]]$ and the left
and right distances $d(A_j,A_{j+1}^T)=d_j$,
$d(A_{j+1}^T,A_j)=\tilde{d}_j$, and the binary codes with the check
matrices $P$ and $P^T$ have distances ${\delta}$ and $\tilde{\delta}$,
respectively.  We have the following upper bounds on the distances in
$K'$:
\begin{theorem}
  \label{th:upper-distance-bnd}
The following upper bounds apply to codes in $\mathcal{B}$: 
  \begin{enumerate}[label=(\alph*)]
  \item If $\tilde{\kappa}\equiv r-u>0$, then
    $d_j'\equiv d(B_j,B_{j+1}^T)\le d_j\equiv d(A_j,A_{j+1}^T)$.
  \item If $k_0\equiv n_0-\rank A_1>0$, then $d_1'\le \delta$.
  \item For $j>1$,
    $d_j'\equiv d(B_j,B_{j+1}^T)\le d_{j-1}\delta\equiv
    d(A_{j-1},A_{j}^T)\,\delta$.
  \end{enumerate}
\end{theorem}
The proofs are based on Lemma \ref{th:rankBj} and its 
generalization, the following Lemma:
\begin{lemma}
  \label{th:rank-3col}
  Consider two pairs of matrices: $A$, $B$ of dimensions
  $n_0\times n_1$, $n_1\times n_2$ respectively, such that $AB=0$, and
  $P$, $Q$ of dimensions $m_0\times m_1$, $m_1\times m_2$
  respectively, such that $PQ=0$.  Then the rank of the following
  block matrix
$$
M=\left(
  \begin{array}[c]{c|c|c} B_{n_1 \times n_2}\otimes E(m_0) &
    E(n_1)\otimes P_{m_0\times m_1} & \\ \hline & A_{n_0\otimes
      n_1}\otimes E(m_1) & E(n_0)\otimes Q_{m_1\times m_2}
  \end{array}
\right)\quad \text{is}
$$
\begin{eqnarray*}
  \rank M&=&m_0\rank B+n_1\rank P-\rank B\rank P \\ 
         & & + m_1\rank A+n_0\rank Q-\rank A\rank Q-\rank A\rank P. 
\label{eq:1}
\end{eqnarray*}
\end{lemma}
In this expression, the first and second three terms are the ranks of
the upper and lower row blocks, respectively; the last term is the
number of rows that enter non-trivial linear combinations between the
two blocks.

\begin{proof}[Proof of Theorem \ref{th:upper-distance-bnd}]
  For convenience, quote the definitions
$$
    B_1=(A_1\otimes E_{r}|E_{n_0}\otimes P),\;
    B_2=\left(
      \begin{array}[c]{c|c}
        A_2\otimes E_r&E_{n_1}\otimes P\\
        & A_1\otimes E_c
      \end{array}\right),\; 
    B_j=\left(
      \begin{array}[c]{c|c}
        A_j\otimes E_r&E_{n_{j-1}}\otimes P\\
        & A_{j-1}\otimes E_c
      \end{array}\right),\ldots 
$$
Part \emph{(a)}: We only need to consider the case where $k_j\neq0$.
In this case we can find a minimum-weight codeword
$\mathbf{c}\in\mathbb{F}_2^{n_j}$, $\wgt(\mathbf{c})=d_j$, such that
$A_j\mathbf{c}^T=0$ but $\mathbf{c}\neq \alpha A_{j+1}^T$ for any
$\mathbf{\alpha}\in\mathbb{F}_2^{n_{j+1}}$.  Clearly,
$\mathbf{x}=(\mathbf{c}\otimes \mathbf{y}|0)$ satisfies
$B_j \mathbf{x}^T=0$ with every $\mathbf{y}\in \mathbb{F}_2^{r}$.  We
want a $\mathbf{y}$ of weight one, such that $\mathbf{x}$ is not a
linear combination of rows of $B_{j+1}^T$.  To see that this is
guaranteed by the condition $\tilde{\kappa}>0$, consider the matrix
$B_{j+1}'$ constructed from the original $A_j$ and a modified matrix
$A_{j+1}'$ of dimension $n_j\times (n_{j+1}+1)$, which is $A_{j+1}$
with an added column $\mathbf{c}^T$; one has
$\rank A_{j+1}'=\rank A_{j+1}+1$.  By construction, $A_jA_{j+1}'=0$.
According to Lemma \ref{th:rankBj}, the modified matrix $B_{j+1}'$ has
$\rank B_{j+1}'=\rank B_{j+1}+r-u$, which guarantees that at least
$\tilde{\kappa}=r-u>0$ of the added columns are indeed linearly
independent from the columns of the original $B_{j+1}$.  That is, we
can find a weight-one vector $\mathbf{y}$ so that $\mathbf{x}$ is a
valid codeword of weight $d_j$ in ${\cal Q}(B_j,B_{j+1})$, which
proves the upper bound.

Proof of Part \emph{(b)} is similar, except now the trial vector has
the form $\mathbf{x}=(0,\mathbf{y}\otimes \mathbf{b})$, where
$P\mathbf{b}^T=0$ (we only need to consider the case $\kappa>0$).  Any
such vector clearly satisfies $B_1\mathbf{x}^T=0$.  We want
$\mathbf{y}\in\mathbb{F}_2^{n_0}$ of unit weight, such that
$\mathbf{x}^T$ be linearly independent from the columns of $B_2$.
Instead of analyzing each possible $\mathbf{y}$, consider the rank of
the matrix $B_2'$ obtained from $B_2$ by adding a block
$E(n_0)\otimes \mathbf{b}^T$ in the second block row (and a zero block
of size $n_{j-1}r\times n_0$ above it).  Resulting matrix $B_2'$
satisfies the conditions of Lemma \ref{th:rank-3col}, which gives
$\rank B_2'=\rank B_2+(n_0-\rank A_1)$; indeed, whenever $k_0>0$, we
can find a codeword of weight $\delta$.

Part \emph{(c)}: Again, we only need to consider the case $k_{j-1}>0$
and $\kappa>0$, so that minimum-weight codewords
$\mathbf{c}\in {\cal Q}(A_{j-1},A_{j}^T)$ and
$\mathbf{b}\in {\cal C}_P^\perp$ cab be found;
$d_{j-1}=\wgt(\mathbf{c})$, $\delta=\wgt(\mathbf{b})$.  The trial
codeword has the block form
$\mathbf{x}=(0|\mathbf{c}\otimes \mathbf{b})$, it satisfies
$B_{j}\mathbf{x}^T=0$.  Also, since $\mathbf{c}^T$ is linearly
independent from the columns of $A_{j-1}$, it follows from Lemma
\ref{th:rank-3col} that $\mathbf{x}^T$ is linearly independent from
the columns of $B_{j+1}$.  Indeed, it is easy to see that if we took
instead of the $\mathbf{c}^T$ a linear combination of the columns of
$A_{j-1}$, the additional block would not affect the rank since it may
be eliminated by column transformations.  If the same were true for
the vector $\mathbf{c}^T$ linearly independent from columns
of $A_{j-1}$, we would get a contradiction with Lemma
\ref{th:rank-3col}.
\end{proof}

Notice that parts (b) and (c) in Theorem \ref{th:upper-distance-bnd}
can be united if we take $d_0=1$ with $k_0>0$.  In the following, we
assume this to be the case.
\begin{theorem}
  \label{th:lower-distance-bnd}
The left minimum distance $d_j'=d(B_j,B_{j+1}^T)$ of the CSS code with
generators $B_j$ and $B_{j+1}^T$ satisfies the following lower bound:
(i) if $\tilde{\kappa}=0$, $d_j'\ge d_{j-1}\delta$. 
(ii) Otherwise, if $\tilde{\kappa}>0$,  $d_j'\ge \min(d_j,d_{j-1}\delta)$.  
\end{theorem}
The proof relies on the following Lemma, a generalization of a
statement used in the proof of the minimum-distance bound for the
hypergraph-product codes\cite{Tillich-Zemor-2009}.


\begin{lemma}
  \label{th:subset-linear-dependence}
  Consider the matrices $B_1$ and $B_2$, such that $B_1B_2=0$,
  constructed from the matrices $A_1^{(n_0\times n_1)}$,
  $A_2^{(n_1\times n_2)}$, and $P^{(r\times c)}$ as in
  Eq.~(\ref{eq:B1}).  Let $I_1\subseteq\{1,2,\ldots,n_1\}$ and
  $I_2\subseteq\{1,2,\ldots,c\}$ denote two arbitrary index sets, and
  the support of each of the vectors
  $\mathbf{a}_i\in \mathbb{F}_2^{n_1}$, $i\le r$, and
  $\mathbf{b}_j\in \mathbb{F}_2^{c}$, $j\le n_0$, respectively, be
  contained inside of $I_1$ and $I_2$.  Consider
  $\mathbf{e}=(\sum_{i=1}^{r} \mathbf{a}_{i}\otimes
  \mathbf{y}_i^{(r)}|\sum_{j=1}^{n_0} \mathbf{y}_j^{(n_0)}\otimes
  \mathbf{b}_j)$ where $\mathbf{y}_i^{(s)}\in\mathbb{F}_2^s$ is a
  vector with the only non-zero element at the position $i$, such that
  $B_1 \mathbf{e}^T=0$.  Consider matrices $B_1'$ and $B_2'$ similarly
  constructed from $A_1'$, $A_2'$, and $P'$, where $A_1'$ and $P'$ are
  constructed from $A_1$ and $P$ by keeping only the columns in $I_1$
  and $I_2$, respectively, and $A_2'$ is a generator matrix of the
  code $\mathcal{C}_{A_2}$ \emph{shortened} to $I_2$ [That is, we take
  a subcode of $\mathcal{C}_{A_2}$ consisting of codewords with
  $c_i=0$ outside of $I_2$, and \emph{puncture} it at these
  positions.]  Also, define vectors $\mathbf{a}_i'$ and
  $\mathbf{b}_j'$ by dropping the (all-zero) components outside of
  $I_1$ and $I_2$, respectively, and the corresponding vector
  $\mathbf{e}'$ which satisfies $G_x'(\mathbf{e}')^T=0$.  With these
  definitions, if $\mathbf{e}'$ is a linear combination of rows of
  $B_2'$, then $\mathbf{e}$ is a linear combination of rows of $B_2$.
\end{lemma}
\begin{proof}[Proof of Theorem \ref{th:lower-distance-bnd}]
  For the reference, we are looking at  
  $$
  B_j=\left(
    \begin{array}[c]{c|c}
      A_j\otimes E_r&E_{n_{j-1}}\otimes P\\ \hline 
                    & A_{j-1}\otimes E_c
    \end{array}\right),\quad 
  B_{j+1}^T=\left(
    \begin{array}[c]{c|c}
      A_{j+1}^T\otimes E_r& \\ \hline 
      E_{n_{j}}\otimes P^T        & A_{j}^T\otimes E_c
    \end{array}\right).
  $$
  Consider a two-block vector
  $\mathbf{e}=(\mathbf{e}_1|\mathbf{e}_2)$, with
  $\mathbf{e}_1\in\mathbb{F}_2^{n_j r}$,
  $\mathbf{e}_2\in\mathbb{F}_2^{n_{j-1} c}$, where
  $w_1\equiv \wgt(\mathbf{e}_1)<d_j$, and
  $w_2\equiv \wgt(\textbf{e}_2)<d_{j-1}\delta$, and assume
  $B_j \mathbf{e}^T=0$.  We are going to show that $\mathbf{e}$ is a
  linear combination of rows of $B_{j+1}^T$ by using Lemma
  \ref{th:subset-linear-dependence} twice.

  step 1: given $\mathbf{e}_1$, mark the columns in $A_j$ which are
  incident on non-zero positions in $\mathbf{e}$.  Denote the
  corresponding index set and the submatrix of $A_j$, respectively, as
  $I_1\subset\{1,2,\ldots,n_j\}$ and $A_j'$.  As in Lemma
  \ref{th:subset-linear-dependence}, denote $A_{j+1}'$ the generator
  matrix of the code $\mathcal{C}_{A_{j+1}}$ shortened at the
  positions outside of $I_1$.  By assumption, $I_1$ is an erasable set
  in $\mathcal{Q}(A_j,A_{j+1}^T)$; this implies that
  $\mathcal{Q}\biglb(A_j',(A_{j+1}')^T\bigrb)$ encodes no qubits.
  Take $P'=P$, and construct the corresponding matrices $B_1'$ and
  $B_2'$, the shortened vectors $\mathbf{a}_i'$, as well as the
  corresponding vector
  $\mathbf{e}'\equiv
  (\mathbf{e}_1'|\mathbf{e}_2)$ which satisfies
  $B_1'(\mathbf{e}_1')^T=0$. 
  The point of the first reduction is that the code
  $\mathcal{Q}'=\mathcal{Q}(A_j',(A_{j+1}')^T)$ encodes no qubits, so
  that the weight of the first block in $\mathbf{e}$ no longer
  matters.

  Step 2: Consider the representation of the vector
  \begin{equation}
  \mathbf{e}_2=\sum_{\ell=1}^c \mathbf{f}_\ell^{(n_{j-1})}\otimes
  \mathbf{y}_\ell^{(c)},\label{eq:second-KP-representation}  
\end{equation}
where the assumed identity $B_{j}(\mathbf{e}')^T=0$ implies
$A_{j-1}\mathbf{f}_\ell^T=0$ for any $1\le \ell\le c$.  For those
$\ell $ where $\mathbf{f}_{\ell}$ is linearly dependent with the rows
of $A_j^T$, $\mathbf{f}_\ell=\alpha_\ell A_j^T$, render this vector to
zero by the linear transformation
$$\mathbf{e}'\to \mathbf{e}'+(0|\alpha_\ell\otimes
\mathbf{y}_\ell^{(c)})\cdot (B_{j+1}')^T.$$ Such a transformation only
affects one vector $\mathbf{f}_\ell$.  The resulting vector
$\bar{\mathbf{e}}'=(\mathbf{e}_1'|\mathbf{e}_2')$ has the second block
of weight $\wgt(\mathbf{e}_2')\le \wgt(\mathbf{e}_2)<d_{j-1}\delta$,
it satisfies $B_{j}'(\bar{\mathbf{e}}')^T=0$, and in the corresponding
block representation (\ref{eq:second-KP-representation}) the remaining
non-zero vectors $\mathbf{f}_\ell^{(n_{j-1})}$ all have weights
$d_{j-1}$ or larger.

This means that, for sure, there remains fewer than $\delta$ of these
non-zero vectors $\mathbf{f}_\ell$.  Therefore, in the representation
$\mathbf{e}_2'=\sum_{j=1}^{n_0}\mathbf{y}_j^{(n_0)}\otimes
\mathbf{b}_j'$ compatible with Lemma
\ref{th:subset-linear-dependence}, the union of supports of vectors
$\mathbf{b}_j'$, $I_2'$, has cardinality $|I_2'|<\delta$.  Indeed,
$I_2'$ is just the set of the indices $\ell$ corresponding to the
remaining non-zero vectors $\mathbf{f}_\ell^{(n_{j-1})}$.

Finally, in step 3, trim the columns of $P$, keeping only the
positions inside $I_2'$.  Since there are fewer then $\delta$ columns
left, $c'=|I_2'|<\delta$, the resulting classical code contains no
non-zero vectors, $c'=\rank P'$.  Now, after we trimmed  the
columns of both $A_j$ and of $P$, according to
Eq.~(\ref{eq:kj-prime}), 
$\mathcal{Q}\biglb( B_j'',\check (B_{j+1})^T\bigrb)$ encodes no qubits; thus the
corresponding vector $\mathbf{e}''$ which satisfies $B_j''
(\mathbf{e}'')^T=0$, is a linear combination of the rows of
$ (B_{j+1}'')^T$.  

We can now use Lemma \ref{th:subset-linear-dependence} to show that
the vector $\bar{\mathbf{e}}'$ is a linear combination of the rows of
$(B_{j+1}')^T$; this remains true for the vector $\mathbf{e}'$.  Using
Lemma \ref{th:subset-linear-dependence} again we see that the original
two-block vector  $\mathbf{e}$ with the block weights $w_1<d_{j}$
and $w_2<d_{j-1}\delta$ which satisfies $B_j\mathbf{e}^T=0$ is
necessarily a linear combination of the rows of $B_{j+1}^T$.  This
guarantees $d_j'\ge \min(d_{j},d_{j-1}\delta)$.  

To complete the proof, consider the case $\tilde{\kappa}=0$
separately.  Here, step 1 can be omitted; the code resulting from
steps 2 and 3 alone would encode no qubits, regardless of the weight
$\wgt(\mathbf{e}_1)$ of the first block.   Thus, in this case we get
the lower bound $d_j'\ge d_{j-1}\delta$.
\end{proof}

\section{Explicit code parametes}

Consider the special case of a four-dimensional construction, from
four matrices $P_j$, $j\in\{1,2,3,4\}$, of dimension $r_j\times c_j$,
with rank $u_j>0$, so that the binary codes with the check matrices
$P_j$ and $P_j'$ have distances $\delta_j$ and $\tilde{\delta}_j$,
respectively.  The parameters of thus constructed codes are given in
the Table \ref{tab:params} on page \pageref{tab:params}

\begin{turnpage}
\begin{table}[htbp]
  \centering\small
  \begin{tabular}[c]{c||c|c|c|c|c}
    $m$ 
    & $n_0^{(m)}=n_0^{(m-1)}r_m$ & $n_1^{(m)}=n_1^{(m-1)}r_m+n_0^{(m-1)}c_m$ & $n_2^{(m)}=n_2^{(m-1)}r_m+n_1^{(m-1)}c_m$ & $n_3$ & $n_4$ \\ \hline
    $1$  & $r_1$ & $c_1$ &  & & \\ \hline 
    $2$  & $r_1r_2$ & $r_1c_2+c_1r_2$ & $c_1c_2$ & & \\ \hline 
    $3$ & $r_1r_2r_3$ & $r_1r_2c_3+r_1c_2r_3+c_1r_2r_3$ 
                    & $r_1c_2c_3+c_1r_2c_3+c_1c_2r_3$ 
                            & $c_1c_2c_3$ & \\ \hline 
    $4$ & $r_1r_2r_3r_4 $ &
                            \begin{minipage}[c]{1.2in}
                              $r_1r_2r_3c_4+r_1r_2c_3r_4$ \\
                              $r_1c_2r_3r_4+c_1r_2r_3r_4$ 
                            \end{minipage} 
    & \begin{minipage}[c]{1.9in}
      $c_1r_2r_3c_4+c_1r_2c_3r_4+c_1c_2r_3r_4$ \\
      $r_1c_2r_3c_4+r_1c_2c_3r_4+r_1r_2c_3c_4$ 
    \end{minipage} 
    &                        \begin{minipage}[c]{1.3in}
                              \strut\hfill$r_1c_2c_3c_4+c_1r_2c_3c_4$ \\
                              $+c_1c_2r_3c_4+c_1c_2c_3r_4$\strut
                            \end{minipage}
    & $c_1c_2c_3c_4$ ${}_{\strut}$ \\ \hline\hline
%  \end{tabular}
%
%  \begin{tabular}[c]{c||c|c|c|c|c}
m     & $k_0^{(m)}=k_0^{(m-1)}\tilde{\kappa}_m$ & $k_1=k_0^{(m-1)}{\kappa}_m+k_1^{(m-1)}\tilde{\kappa}_m$ & $k_2$ & $k_3$ & $k_4$ \\ \hline
 1& $\tilde{\kappa}_1 $
            & $\kappa_1$ & &     \\    \hline   
    2  & $\tilde{\kappa}_1\tilde{\kappa}_2 $
                    & $\tilde{\kappa}_1\kappa_2  +\kappa_1\tilde{\kappa}_2$
                            & $\kappa_1\kappa_2$ &  & \\ \hline 
    3& $\tilde{\kappa}_1 \tilde{\kappa}_2 \tilde{\kappa}_3$ 
            & ${\kappa}_1 \tilde{\kappa}_2
              \tilde{\kappa}_3+\tilde{\kappa}_1 {\kappa}_2
              \tilde{\kappa}_3+\tilde{\kappa}_1 \tilde{\kappa}_2
              {\kappa}_3$
                    & 
                      ${\kappa}_1 {\kappa}_2
                      \tilde{\kappa}_3+{\kappa}_1 \tilde{\kappa}_2 
              {\kappa}_3+\tilde{\kappa}_1 {\kappa}_2
              {\kappa}_3$
            &  ${\kappa}_1{\kappa}_2  {\kappa}_3$\\ \hline 
    4 &
        $\tilde{\kappa}_1\tilde{\kappa}_2\tilde{\kappa}_3\tilde{\kappa}_4$
            & \begin{minipage}[c]{1.5in}
              \strut\hfill
              ${\kappa}_1\tilde{\kappa}_2\tilde{\kappa}_3\tilde{\kappa}_4+\tilde{\kappa}_1{\kappa}_2\tilde{\kappa}_3\tilde{\kappa}_4$\\
              $+\tilde{\kappa}_1\tilde{\kappa}_2{\kappa}_3\tilde{\kappa}_4+\tilde{\kappa}_1\tilde{\kappa}_2\tilde{\kappa}_3{\kappa}_4$\strut
            \end{minipage}
            & \begin{minipage}[c]{2.3in}
              \strut\hfill
              ${\kappa}_1{\kappa}_2\tilde{\kappa}_3\tilde{\kappa}_4+{\kappa}_1\tilde{\kappa}_2{\kappa}_3\tilde{\kappa}_4+{\kappa}_1\tilde{\kappa}_2\tilde{\kappa}_3{\kappa}_4$\\
              $+\tilde{\kappa}_1{\kappa}_2{\kappa}_3\tilde{\kappa}_4+\tilde{\kappa}_1{\kappa}_2\tilde{\kappa}_3{\kappa}_4+\tilde{\kappa}_1\tilde{\kappa}_2{\kappa}_3{\kappa}_4$\strut
            \end{minipage} 
    & \begin{minipage}[c]{1.5in}
      \strut\hfill
      ${\kappa}_1{\kappa}_2{\kappa}_3\tilde{\kappa}_4+{\kappa}_1{\kappa}_2\tilde{\kappa}_3{\kappa}_4$\\
      $+{\kappa}_1\tilde{\kappa}_2{\kappa}_3{\kappa}_4+\tilde{\kappa}_1{\kappa}_2{\kappa}_3{\kappa}_4$\strut
            \end{minipage} & ${\kappa}_1{\kappa}_2{\kappa}_3{\kappa}_4$
\\ \hline\hline 
    & $d_0$ & $d_1$ & $d_2$ & $d_3$ & $d_4$ \\ \hline 
1 & $\tilde{\kappa}_1>0? 1:\infty$ & $\delta_1$ & & &  \\ \hline 
2 & $\tilde{\kappa}_2>0? d_0^{(m=1)}:\infty$
                                 &   \begin{minipage}[c]{2.5in}
                   $\tilde{\kappa}_2>0?
                   \min\biglb(d_1^{(1)},d_0^{(1)}\delta_2\bigrb):d_0^{(1)}\delta_2
                   {}_{\strut}^{\strut}$\\
                   \hrule{\hfill}                 
                   $\tilde{\kappa}_1\tilde{\kappa}_2>0:  \;  
                   \min({\delta}_1,\delta_2)$\hfill\strut\\
                   $\tilde{\kappa}_1=0:  \;
                   \delta_2$\hfill\strut\\
                   $\tilde{\kappa}_2=0:  \;
                   \delta_1$\hfill\strut
                 \end{minipage}
    & $\delta_1\delta_2$ &  \\ \hline 
    3&  $\tilde{\kappa}_3>0? d_0^{(m=2)}:\infty$ &   \begin{minipage}[c]{2.2in}
                   $\tilde{\kappa}_3>0:  \;  
                   \min(d_1^{(m=2)},d_0^{(m=2)}\delta_3)$\hfill\strut\\
                   $\tilde{\kappa}_3=0:  \;
                   d_0^{(m=2)}\delta_3$\hfill\strut
                 \end{minipage} 
    &   \begin{minipage}[c]{2.1in}
      $\tilde{\kappa}_3>0:  \;  
      \min(d_2^{(m=2)},d_1^{(m=2)}\delta_3)$\hfill\strut\\
      $\tilde{\kappa}_3=0:  \;
      d_1^{(m=2)}\delta_3$\hfill\strut
    \end{minipage} & $\delta_1\delta_2\delta_3$
\\ \hline 
    4 & $\tilde{\kappa}_4>0? d_0^{(m=3)}:\infty$ &  \begin{minipage}[c]{2.1in}
      $\tilde{\kappa}_4>0:  \;  
      \min(d_1^{(m=3)},d_0^{(m=3)}\delta_4)$\hfill\strut\\
      $\tilde{\kappa}_4=0:  \;
      d_0^{(m=3)}\delta_4$\hfill\strut
    \end{minipage} & \ldots & \ldots & $\delta_1\delta_2\delta_3\delta_4$
  \end{tabular}
  \caption{Parameters of the first four chain complexes.  The distance
    is minimized over all
    non-zero combinations of the products of $\delta_j$ and
    $\tilde{\delta}_j$ corresponding to the products of $\kappa_j$ and $\tilde{\kappa}_j$
    that actually contribute to the total $k$.}
  \label{tab:params}
\end{table}
\end{turnpage}


\section{row and column weight of CSS code}
In many quantum error models, the probability of measurement error is related to the number of gate operations or qubits involved in each measurement. The row and column weight of the matrices in the complex chain can be used to determine upper and lower bound of the weight of check operators, and so the measurement error probability. It can easily be derived from the parameters of $P$.

\begin{theorem}
Let $\omega_j$ be the maximum weight of rows in $P_j$, and $\tilde\omega_j$ be the maximum weight of columns in $P_j$.
For matrix $A^{(m)}_j$ in the $m$-chain complex, the maximum row weight is $\displaystyle \omega^{(m)}_j=\max_{\{I^{(m-j+1)}\}} S^{(m)}(\{I^{(m-j+1)}\})$, and the maximum column weight is
$\displaystyle \tilde\omega^{(m)}_j=\max_{\{I^{(j)}\}} \tilde S^{(m)}(\{I^{(j)}\}) $. The $j$-term summation is given by $\displaystyle  S^{(m)}(\{I^{(j)}\}) = \sum_{i=1}^{j}  \omega^{(m)}_{I^{(j)}_i}$, $\displaystyle \tilde S^{(m)}(\{I^{(j)}\}) = \sum_{i=1}^{j} \tilde \omega^{(m)}_{I^{(j)}_i}$, where $\{I^{(j)}\}$ is an arbitrary size-$j$ subset of the index set $\{1,2,...,m\}$

\end{theorem}
\begin{proof}
The maximun row weight of $A^{(m)}_j$ would be  $\omega^{(m)}_j=\max( \omega^{(m-1)}_j+\omega_m,\omega^{(m-1)}_{j-1})$, and the maximun column weight $\tilde\omega^{(m)}_j=\max( \tilde\omega^{(m-1)}_{j-1}+\tilde\omega_m,\tilde\omega^{(m-1)}_{j})$ for $j=1,...,m$. Let $\omega^{(m)}_0=\omega^{(m)}_{m+1}=\tilde\omega^{(m)}_0=\tilde\omega^{(m)}_{m+1}=0$ to complete the rules.
Apply the iteration, one can get the general expresion.
\end{proof}

For the CSS code $ {{\cal Q}^{(m)}}(A_{j},A_{j+1}^T)$, the max row weight is \framebox{ $\max( \omega^{(m)}_j, \tilde \omega^{(m)}_{j+1})$}, which is the upper bound on weight of check operators. And the max column weight is $\max( \tilde\omega^{(m)}_j,  \omega^{(m)}_{j+1})$.

In order to find the min weight $\omega'$, replace $\max$ with $\min$ and the same rules will apply. The difference is, when finding max weight, combination of rows or columns are not allowed. But when finding min weight, one can combine row or columns to get smaller weight. The reason is that we use the former to find weight but use the latter to determine the distance.


\begin{table}[htbp]
  \centering\small
  \begin{tabular}[c]{c||c|c|c|c}
  row weight & $\omega^{(m)}_1$ & $\omega^{(m)}_2$ & $\omega^{(m)}_3$ & $\omega^{(m)}_4$ \\\hline
  1& $ \omega_1$ &&\\\hline
  2&   $\omega_1 + \omega_2$ & $\max(\omega_1,\omega_2)$ && \\\hline
  3& $\omega_1+\omega_2+\omega_3$ 
  & $\max(\omega_1 \omega_2, \omega_1 \omega_3, \omega_2 \omega_3)$
  & $\max(\omega_1,\omega_2,\omega_3)$ \\\hline
  4& $\omega_1+\omega_2+\omega_3+\omega_4$
  & 
  \begin{minipage}[c]{1.2in}
  $\max(\omega_1+\omega_2+\omega_4,$
  $\omega_1+\omega_3+\omega_4,$
$  \omega_2+\omega_3+\omega_4,$
$\omega_1+\omega_2+\omega_3)$
                            \end{minipage} 
  &  \begin{minipage}[c]{1.4in}
  $\max(\omega_1+\omega_2,\omega_1+\omega_3,$
  $\omega_2+\omega_3,
  \omega_1+\omega_4,$
  $\omega_2+\omega_4,\omega_3+\omega_4)$
     \end{minipage} 
  &$\max(\omega_1,\omega_2,\omega_3,\omega_4)$
                              \\\hline\hline

    col weight & $\tilde\omega^{(m)}_1$ & $\tilde\omega^{(m)}_2$ & $\tilde\omega^{(m)}_3$ & $\tilde\omega^{(m)}_4$ \\\hline
  1& $ \tilde\omega_1$ &&\\\hline
  2& $\max(\tilde\omega_1,\tilde\omega_2)$ 
  & $\tilde\omega_1 + \tilde\omega_2$ && \\\hline
  3  & $\max(\tilde\omega_1,\tilde\omega_2,\tilde\omega_3)$ 
  & $\max(\tilde\omega_1 \tilde\omega_2, \tilde\omega_1 \tilde\omega_3, 
  \tilde\omega_2 \tilde\omega_3)$
  & $\tilde\omega_1+\tilde\omega_2+\tilde\omega_3$ \\\hline
  4&$\max(\tilde\omega_1,\tilde\omega_2,\tilde\omega_3,\tilde\omega_4)$
  &   \begin{minipage}[c]{1.4in}
  $\max(\tilde\omega_1+\tilde\omega_2,\tilde\omega_1+\tilde\omega_3,$
  $\tilde\omega_2+\tilde\omega_3,
  \tilde\omega_1+\tilde\omega_4,$
  $\tilde\omega_2+\tilde\omega_4,\tilde\omega_3+\tilde\omega_4)$
     \end{minipage} 
     &\begin{minipage}[c]{1.2in}
  $\max(\tilde\omega_1+\tilde\omega_2+\tilde\omega_4,$
  $\tilde\omega_1+\tilde\omega_3+\tilde\omega_4,$
$ \tilde \omega_2+\tilde\omega_3+\tilde\omega_4,$
$\tilde\omega_1+\tilde\omega_2+\tilde\omega_3)$
                            \end{minipage} 
  & $\tilde\omega_1+\tilde\omega_2+\tilde\omega_3+\tilde\omega_4$
                              \\\hline\hline

\end{tabular}
  \caption{upper bound on row and column weight of matrices $A^{(m)}_j$ in the chain complex. $\omega_i$ and $\tilde\omega_i$ are the row and column weight of $P_i$.  To get the lower bound $\omega'^{(m)}_i$ and $\tilde\omega'^{(m)}_i$, just replace $\max$ by $\min$ (I think this statement about min weight is wrong)
  }
  \label{tab:weight}
\end{table}


\section{lower bound on distance of Data syndrome code}
From a quantum code with redundant parity check matrix $G_{n_1 \times n_2}$, we define the data syndrome code as $Q=( G|E_{n_1} )$. There exists full rank matrix $T_{\kappa \times n_1}$ such that $TG=0$. $T$ is the parity check matrix for the classical syndrome code. Then the generating matrix for syndrome code is $F$ such that $TF^T=0$. One can see $F$ is just the full rank matrix of $G^T$. Then the minimum row weight of $F$ is just the minimum weight of $G^T$, which is the distance of the syndrome code $d_s$. Then $[n_1,k,d_s]$ is the parameters of syndrome code with parity check matrix $T$. $k=n_1-\kappa$ is the rank of $G$.

Apply it to code $ {\cal Q}^{(m)}(A_{j},A_{j+1}^T)$, I will get \framebox{$[n^{(m)}_{j-1},\rank(A_j),\tilde\omega'^{(m)}_j]$} for the left part, here prime mean minimum instead of maximum. For the right part, the syndrome code has parameters  \framebox{$[n^{(m)}_{j+1},\rank(A_{j+1}),\omega'^{(m)}_{j+1}]$}. Combine them, I will get  \framebox{$[n^{(m)}_{j-1}+n^{(m)}_{j+1}, n^{(m)}_j-k^{(m)}_j ,\min(\tilde \omega'^{(m)}_j,\omega'^{(m)}_{j+1})]$}

To avoid confusion in my notation, the data syndrome code means one decode qubit error and syndrome error globally/simultaneously. If one decode syndrome error first, and then use the decoded syndrome to decode the qubit error,  the syndrome part is called syndrome code, and the qubit part is the original quantum code.

\begin{theorem}
 (1) If the quantum code has distance $d_q$, and the syndrome code has distance $d_s$, then the overall code (decode syndrome error first and then qubit error) would have distance \framebox{$d=\min(d_q,d_s)$}.

(2) If we decode the qubit error and syndrome error simultaneously, the data syndrome code would have distance 
\framebox{ $d_{ds}=\min \wgt\left( (E_{n_2}|G^T) \setminus (H|0)   \right)\ge \min(d_q,1+d_s)$}, where $H$ is the other parity check matrix.
\end{theorem}
\begin{proof}
 Part (1) looks obvious. 

For part (2), the dual matrix of $Q=( G|E_{n_1} )$ is $(E_{n_2}|G^T)$. We can transform it into $ \left( \begin{array}{cc} H & 0 \\ C & 0\\ A &B \end{array} \right)$, where $GH^T=0$, $GC^T =0$, $H$ is the other parity check matrix, $C$ is the codeword generating matrix, $A$ and $B$ are full rank matrices, which refer to detectable qubit errors and corresponding syndrome in the original quantum code. The distance of the data syndrome code is the minimum row weight of $(H|0)$ 's coset in $ \left( \begin{array}{cc}  C & 0\\ A &B \end{array} \right)$. We can write the matrix as $ (E|G^T) \setminus (H|0)   $. 
Use $\min \wgt (C|0)=\min \wgt (C)=d_q$ and
$$\min \wgt (A|B) = \min \wgt \left( (E|G^T) 
\setminus \left( \begin{array}{cc} H & 0 \\ C & 0 \end{array} \right) \right)
\ge 1+ \min \wgt (G^T) =1+d_s$$
The we have the following lower bound
$$d_{ds} =\min \left(   d_q, \min \wgt (A|B)  \right)
\ge \min(d_q,1+d_s)$$
\end{proof}
Example: For the cubic code with side length $L$, the $G_X$ with plaquette check operators has $d_q=L^2$ and $d_s=4$. In part (1), it will lower the overall distance to 4. In part (2), the distance becomes 5. Both match the lower bounds exactly. And both become very weak, compared to the quantum code distance.
Then this kind of code only apply to the case where measurement error probability are much lower than quantum error probability.

For the $(A|B)$ part, we consider it as bad/logical errors. But the measurement error will disappear in the next round and then the qubit error can get detected. Then $d_{ds}$ would be higher than the lower bound especially when $d_q \gg d_s$.

Then we can get a better lower bound by looking at the row weight distribution of $(E|G^T)$

\begin{definition}[robustness] 
define $g(x)$ as the min weight of error $e_q$ $s.t.$ $|\sigma (e_q) |$ = x. We call it robustness. Then 
$\min \wgt \left( (E|G^T) 
\setminus \left( \begin{array}{cc} H & 0 \\ C & 0 \end{array} \right) \right)
=\min_x (x+g(x))=d_r$, \framebox{ $d_{ds}=\min ( d_q,d_r)$}
\end{definition}
In order to get large distance, we want $g(x)$ to be large. Also, we prefer $g(x)$ to have an inverse relation with $x$, such as $g(x)=d_q-x$.
\textbf{ This gives a standard to choose a set of parity check generators, that maximize the distance of the data syndrome code.} Fujiwara \cite{fujiwara2014ability} gives several examples of this. Increasing robustness $g(x)$ is the generalization of those examples, and is a special method to increase the column weight of $G$. I believe we can design an optimization algorithm by calculating the row weight distribution of $G^T$, to make the code have best distance. The output would be a transformation matrix for the rows of $G$.

As for the $soundness$ defined by Earl, he wants make $g(x)$ as small as possible, to avoid large weight residual errors resulted from measurement error. This applies in the case that one decode syndrome error first and then qubit error. This reduces the decoding complexity.

But If we do a global decoding for syndrome error and qubit error together, It is not necessary to have good soundness. Conversely, large minimum weight are prefered to increase the distance.

As Earl suggested, if the residual error has weight $\wgt(e_{res})<d_q/2$, we consider it as a good error, which can hopefully get fixed in the next round. Then we can have a modified definition of the robustness, which increases the lower bound compare to the previous definition.

\begin{definition}[modified robustness] 
define $g'(x)$ as the min weight of error $e_q$ $s.t.$ $|\sigma (e_q) |$ = x and $\wgt(e_q) \ge d_q/2$. If such error doesn't exist, then $g(x)=\infty$.
Then 
$\min \wgt \left( (E|G^T) 
\setminus \left( \begin{array}{cc} H & 0 \\ C & 0 \end{array} \right) \right)
=\min_x (x+g'(x))=d'_r \ge 1+d_q/2$, $d_{ds}=\min ( d_q,d'_r) \ge 1+d_q/2$
\end{definition}
This modified robustness can change the story of this work, and is out of my mind right now. In my imagination, the good soundness is related to a case that, for some $t$, $g'(x)=\infty, \forall x \le t$

\section{Generalization to non CSS code}
This section generalizes the robustness theory to non CSS code, and show an example of optimizing the $[[7,1,3]]$ CSS code.
For a general code with parity check matrix $G=(G_X|G_Z)$, $\tilde G G^T=0$. $C$ is the codeword generating matrix $\tilde G C^T=0$. Define the data syndrome code as $D=(G_X|G_Z|E)$. An error $e=(e_X|e_Z|e_s)$ is a codeword iff $D(e_Z|e_X|e_s)^T=0$. The dual matrix of $D$ is  
$J=\left( \begin{array}{cc} E & \begin{minipage}[c]{0.22in} $G_X^T$\\$G_Z^T$ \end{minipage}  \end{array} \right) $. 
The codeword generating matrix of $D$ is $J \setminus \left( \begin{array}{ccc}C_X & C_Z & 0 \\ G_X & G_Z & 0 \end{array} \right)$. Usually, the row weight (supposition of $X$ and $Z$ error on a qubit is a $Y$ error of weight 1) of the codeword generating matrix give the distance of the code. However, other than X and Z error, we also need to consider Y errors. In this matrix, an X error and Z error on the same qubits will have weight two, but it is actually an Y error of weight one.
To avoid this, the matrix used to determine the distance becomes 
$\left( \begin{array}{cc} E & \begin{minipage}[c]{0.22in} $G_X^T$\\$G_Z^T$\\$G_Y^T$ \end{minipage}  \end{array} \right) 
 \setminus \left( \begin{array}{cccc}C_X & C_Z & C_Y &0 \\ G_X & G_Z & G_Y&0 \end{array} \right)$. In CSS code, the Y part could be eliminated casue its weight is always larger than the X or Z part. I didn't show Y part in the examples, but it should be checked and was checked by hand.



Fujiwara modify the $[[7,1,3]]$ CSS code to make it a non CSS code which can fix single measurement error\cite{fujiwara2014ability} . The parameters of the quantum code doesn't change. The original CSS code has parity check matrix

$$(G_X|G_Z)=\left(
\begin{array}{ccccccc|ccccccc}
1&0&0&1&0&1&1\\
0&1&0&1&1&0&1\\
0&0&1&0&1&1&1\\
&&&&&&&1&0&0&1&0&1&1\\
&&&&&&&0&1&0&1&1&0&1\\
&&&&&&&0&0&1&0&1&1&1\\
\end{array}
\right)$$
Transpose of the dual matrix is
$$
\left( \begin{array}{cc} E & \begin{minipage}[c]{0.22in} $G_X^T$\\$G_Z^T$ \end{minipage}  \end{array} \right) ^T=
\left(
\begin{array}{ccccccc|ccccccc}
1&0&0&1&0&1&1\\
0&1&0&1&1&0&1\\
0&0&1&0&1&1&1\\
&&&&&&&1&0&0&1&0&1&1\\
&&&&&&&0&1&0&1&1&0&1\\
&&&&&&&0&0&1&0&1&1&1\\\hline
1&&&&&&&\\
&1&&&&&&&\\
&&1&&&&&&&\\
&&&1&&&&&&&\\
&&&&1&&&&&&&\\
&&&&&1&&&&&&&\\
&&&&&&1&&&&&&&\\\hline
&&&&&&&1\\
&&&&&&&&1\\
&&&&&&&&&1\\
&&&&&&&&&&1\\
&&&&&&&&&&&1\\
&&&&&&&&&&&&1\\
&&&&&&&&&&&&&1\\
\end{array}
\right)$$
Eliminating the parity check operators, the min column weight is 2. Then the distance of the syndrome code is $d=2$. It is not able to fix single measurement error.

Fujiwara apply the recombination of parity check operators, the paritycheck matrix becomes
$$(G'_X|G'_Z)=\left(
\begin{array}{ccccccc|ccccccc}
1&0&0&1&0&1&1 &1&0&0&1&0&1&1\\
0&1&0&1&1&0&1 &1&0&0&1&0&1&1\\
0&0&1&0&1&1&1 &1&0&0&1&0&1&1\\
1&1&1&0&0&0&1 &0&1&1&1&0&1&0\\
1&1&1&0&0&0&1 &1&0&1&1&1&0&0\\
1&1&1&0&0&0&1 &1&1&0&0&1&1&0\\
\end{array}
\right)$$
Then the transpose of its dual becomes
$$
\left( \begin{array}{cc} E & \begin{minipage}[c]{0.22in} $G'^T_X$\\$G'^T_Z$ \end{minipage}  \end{array} \right) ^T=
\left(
\begin{array}{ccccccc|ccccccc}
1&0&0&1&0&1&1 &1&0&0&1&0&1&1\\
0&1&0&1&1&0&1 &1&0&0&1&0&1&1\\
0&0&1&0&1&1&1 &1&0&0&1&0&1&1\\
1&1&1&0&0&0&1 &0&1&1&1&0&1&0\\
1&1&1&0&0&0&1 &1&0&1&1&1&0&0\\
1&1&1&0&0&0&1 &1&1&0&0&1&1&0\\\hline
1&&&&&&&\\
&1&&&&&&&\\
&&1&&&&&&&\\
&&&1&&&&&&&\\
&&&&1&&&&&&&\\
&&&&&1&&&&&&&\\
&&&&&&1&&&&&&&\\\hline
&&&&&&&1\\
&&&&&&&&1\\
&&&&&&&&&1\\
&&&&&&&&&&1\\
&&&&&&&&&&&1\\
&&&&&&&&&&&&1\\
&&&&&&&&&&&&&1\\
\end{array}
\right)$$
Eliminating the parity check operators, the min column weight is 3. Then the distance of the syndrome code is $d=3$. It is able to fix all single measurement error.

To see the change from the view of robustness, lets calculate it for both codes.

For the original CSS code, the first 3 columns show there exists a single qubit error for that syndrome bit. So $g(1)=1$.  The column 4 show that we get a single qubit error corresponding to 2 syndrome bit, then $g(2)=1$, etc. Then $d=1+g(1)=2$. 

For the modified code, adding column 1, 2, 7, we got $g'(1)=3$. The column 4 shows $g'(1)=2$. Column 4 and column 5 show that $g'(2)=1$, etc. Then $d'=1+g'(1)=2+g'(2)=3$.

Fujiwara optimize the code in a way that all single errors (both qubit and syndrome bit) has unique syndrome result. In our language, we are improving the robustness $\min_x (x+g(x)$ from 2 to 3. When designing program, this could be achieved by applying random/particular transform matrix on $G$ and evaluate the robustness. I think there is a way to design such transform matrix, in order to eliminate those columns that give the minimum value of $(x+g(x))$


\section{Examples: Repetition code and 2D toric code}
For the 1D repetition code without redundancy, the parity check matrix is
$\hat a=\left( \begin{array}{ccccc}1 & 1 \\  & 1 & 1 & \\ &&1 &1\\ &&&1&1\end{array} \right)$, the first and last column gives $g(1)=1$, and other columns give $g(2)=1$, which implement $d_{ds}=\min_x(x+g(x))=2$. If we add one redundancy to make it the circulant repetition matrix $a=\left( \begin{array}{ccccc}1 & 1 \\  & 1 & 1 & \\ &&1 &1\\ &&&1&1\\ 1 &&&&1\end{array} \right)$, then we still have $g(2)=1$, but $g(1)=\infty$. Now $d_{ds}=3$.

Similarly, for the 2D toric code without redundancy, for most single qubit error, there are two plaquettes connected, then $g(2)=1$. But for the one plaquette removed by redundancy, the single qubit error on one of the bonds of that plaquette is connected with only one plaquette. Then $g(1)=1$. Hence $d_{ds}=2$. But if we add that removed plaquette, then all single error are connected with two plaquettes. Then $g(2)=1$, $g(1)=\infty$, and $d_{ds}=3$. Same conclusion applied to the reduandancy in toric codes in higher dimension.

In these two cases, we add extra checks to increase $d_{ds}$. But in some cases, we only need to do recombination of checks. For example
$a'=\left( \begin{array}{ccccc}1 & 1 \\  & 1 & 1 & \\ &1& &1\\ &&&1&1\\ 1 &&&&1\end{array} \right)$, where I add row 2 and row 3 in the circulant repetition matrix $a$. Column 3 in $a'$ shows $g(1)=1$ and $d_{ds}=2$. By converting it back to code $a$, the distance get improved.

\textbf{These examples show that, if a data syndrome code has only a small number of codewords that have weight matches $d_{ds}$, then we may eliminate those codewords and increase $d_{ds}$, by paying a small cost on increasing redundancy of parity checks and/or weight of parity checks. This case maybe popular when we construct hypergraph product code from random matrices. Another advantage for the hypergraph product code is that its row weight is bounded}

For the 2D toric code with redundancy that have $d_{ds}=3$, since all single qubit error connected with two plaquette match this minimum weight, it is much harder to increase $d_{ds}$. Lets try an example. Let\\
$b=a^T=\left( \begin{array}{ccccc} 1&&&&1 \\1 & 1 \\  & 1 & 1 & \\ &&1 &1\\ &&&1&1 \\\end{array} \right)$, 
$T=\left( \begin{array}{ccccc} 1 & 1 \\  & 1 & 1 & \\ &&1 &1\\ &&&1&1 \\ 1&&&&1 \\1\end{array} \right)$.\\
The 2D toric code with one redundancy is defined as 
$G=(a \otimes E |e \otimes b) $.
Then define a new code
$G'=\left( \begin{array}{c} (T \otimes E) G \\ (E \otimes T) G\end{array}\right)  $, where we combine any two adjacent plaquettes to form a rectangular check operator. The top matrix means horizontal rectangles, and the bottom matrix means vertical rectangles. (The last row in $T$ ensures that we are not changing the quantum code) Now, each single bond is connected to 6 rectangle, $g(6)=1$. Each two bonds could connect to 8 rectangles, $g(8)=2$. Then $d_{ds}=7$. Here we double the number of parity checks. 
As a comparison, if we just repeat the measurement twice, then a single qubit error will be connected to 4 plaquettes. Hence $g(4)=1$ and $d_{ds}=5$.

Above examples are giving by handy construction or topological graphs. I want to give another example, where we purely look at the weight of the matrix and apply the robustness optimization method. Then we come back to the $[[7,1,3]]$ code. This time we only allow row combination but  don't add extra rows. The column 1, 2, 3 in the parity check matrix has weight 1, which we want to eliminate. By repeating the first three rows, the first three column will have weight 2. But this will leave no room for recombination of the Z operators.
Hence, we add all pairs of rows in the first three rows, that is, we apply 
$\left( \begin{array}{ccc} 1\\&1\\&&1\\1&1\\&1&1\\1&&1 \end{array} \right)$ to the first three rows. Then we do the same thing for the last three rows since Z checks and X checks are identical. That is, we apply 
$T = \left( \begin{array}{ccc|ccc} 1&&&1&1\\&1&&&1&1\\&&1&1&&1\\1&1&&1\\&1&1&&1\\1&&1&&1 \end{array} \right) = \left( \begin{array}{cc} E & a \\a & E \end{array} \right)$ to parity check matrix $(G_X|G_Z)$, and get
$$T(G_X|G_Z)=\left(
\begin{array}{ccccccc|ccccccc}
1&0&0&1&0&1&1 &1&1&0&0&1&1&0\\
0&1&0&1&1&0&1 &0&1&1&1&0&1&0\\
0&0&1&0&1&1&1 &1&0&1&1&1&0&0\\
1&1&0&0&1&1&0 &1&0&0&1&0&1&1\\
0&1&1&1&0&1&0 &0&1&0&1&1&0&1\\
1&0&1&1&1&0&0 &0&0&1&0&1&1&1\\
\end{array}
\right)$$
Now min column weight is 3. we also need to check Y part
$$G_Y=G_x+G_Z=\left(\begin{array}{ccccccc}
0&1&0&1&1&0&1\\
0&0&1&0&1&1&1\\
1&0&0&1&0&1&1\\
0&1&0&1&1&0&1\\
0&0&1&0&1&1&1\\
1&0&0&1&0&1&1\\
\end{array}\right)$$
The min column weight is 2. Then $g(2)=1$, $d_{ds} =3$, it is a single error correction code.

Above examples are based on robustness. If we consider the modified robustness, things will become quite different: we are changing the definition of distance of data syndrome code. First, the distance of cubic code is not 5 anymore, cause we can ignore codeword involved with single qubit error.

\section{Upper bound on distance of data syndrome code}
In Fujiwara's paper, he optimize a data syndrome code to be robust against single error, by ensuring each single error has a unique syndrome. He also appply the same technique for double error correction codes. from this, I can give an upper bound on the distance of data syndrome code.
\begin{theorem}
For a data syndrome code with $n$ qubits and $m$ number of parity checks. In order to correct all single error, including X,Y, Z qubit error and syndrome bit error. the following relation should be satisfied
$3\left( \begin{array}{c} n\\1 \end{array} \right)+m \le 2^m$
\end{theorem}
\begin{proof}
This is the necessary condition that each single error has a unique syndrome.
\end{proof}
\begin{theorem}
For a data syndrome code with $n$ qubits and $m$ number of parity checks. In order to correct all error $e$, $s.t.$ $\wgt{e}\le t$, including X,Y, Z qubit error and syndrome bit error. the following relation should be satisfied
$\displaystyle \sum_{i=1}^t  \sum_{j=0}^i 3^j\left( \begin{array}{c} n\\j \end{array} \right) \times \left( \begin{array}{c} m\\i-j \end{array} \right)  \le 2^m$
\end{theorem}
\begin{proof}
This is the necessary condition that all such errors has a unique syndrome.
\end{proof}
This upper bound shows that, for some code, it is unable to improve $d_{ds}$ without adding new redundant checks. An example is the case of perfect five qubit code in Fujiwara's paper. An extra check has to be added to make it robust against all single error.
These two bounds may be equivalent to the theorem in Fujiwara's paper on single error correction codes and double error correction codes.

\section{discussion}
After we have these upper and lower bound, and the robustness theory, we can apply it to hypergraph product code and also, the quantum convolutional code design with Alexei Ashikmin. Notice for the hypergraph product codes, if we want to bound its weight, then we can only apply transformation of small matrices $P_i$ other than the whole parity check matrix.

The numerical method to evaluate robustness is simple, just estimate weight of all codewords with $x=1,2,...,d_{ds}-1$. But its complexity increase exponentially with $x$. If we want to design code with large distance, optimal evaluation method need to be find. Random window decoder is one of the candidates. To optimize robustness, one can apply random transformation matrix and evaluate robustness. But it is better to find constraints on such transformation matrices. 

As I am trying to maximize the robustness, in Earl's paper he want to minimize similar quantity soundness. I think what he get is small weight check operators and small decoding complexity. And my optimization is good distance.

The last thing is the modified robustness, which may be more realistic for repeating measurement than the robustness.

\bibliography{lpp,qc_all,more_qc,linalg,WeileiBibFile}
\end{document}
